\documentclass{article}

% For custom margins
\usepackage{anysize}

\marginsize{2cm}{2cm}{2cm}{2cm}

\title{Company: Allion Test Labs\\
Intern: Russell Miller\\
Mentor: Chad Meyer\\
Mentor Phone: 509-995-4444\\
Mentor E-mail: chadmeyer@us.allion.com\\
Discipline: Computer Science\\
Internship Level: Junior Year\\
\today}
\date{}

\begin{document}

\maketitle

\pagebreak

% Description of the company and interns role in the company. Identify intern’s
% supervisor and the department and/or division where they were employed.
\section*{About Allion}
Allion was established in 1991. Through custom test plans and automation, Allion
offers engineering services for performance and compliance testing. 
The specialty of the lab in Beaverton is advanced Wi-Fi testing. For performance
testing, there is an anechoic chamber and other equipment that allows isolation
of signal. Using attenuation simulators, several use cases can be tested 
in-depth.

\section*{My Role at Allion}
Projects at Allion coincide with contracts from third parties. Often these
projects involve testing a new product from those companies. I worked on a
couple of those projects, but worked on side projects as well when we were
between contracts. I developed automation for the attenuation equipment and
wrote full tests that used that automation. My mentor Chad was extremely 
helpful, always willing to teach me about wireless technology and testing.\\
\\
Rather than having departments Allion is split into teams per project. When I
first started my internship I started out helping analyze results from the last 
round of testing for the Nook Color, when it was still pre-release. When that
project ended I did independent work until another testing contract started
later on in my internship.

\section*{The Projects I Participated In}
\begin{itemize}
\item New product testing for Barnes \& Noble's Nook Color
\item Writing a Python library for the Azimuth Adept-N
\item Writing a Python library for the Azimuth Ace
\item Writing automated testing which employs the Adept-N and Ace
\item Super top secret new product testing for a new startup company
\item Super top secret new product testing for Amazon
\end{itemize}

\section*{Executive Summary}
On the Nook Color, I was doing an investigation of the data we gathered,
attempting to provide the value-add of knowing why certain tests failed.
The utility and testing libraries I wrote involved doing research on the
equipment's API, gathering requirements from my mentor and engineers on site,
maintaining these libraries over the course of the internship, and writing
documentation for engineers to allow them to run my scripts with ease.
With the secret projects where we tested new products, I was able to write the
test methods to meet the needs of the test plan we had provided to the company
that was developing the new product. I wrote several scripts and ran them over
a large number of iterations, as the test plan called for. There was a team of
engineers using the scripts I wrote, so I also had to show them how it worked.\\
\\
The chief export of Allion is the ability to produce the tests outlined in a
test plan, which is agreed upon by the company paying for the contract. On a
regular basis, we keep the customer updated on completion of the test plan, and
we were able to impress our customers with timely results and added value of
debugging problematic scenarios. On one of our secret projects we were given the
opportunity to add bug reports to their database, and I was able to benefit
not only the customer but the face of Allion by paying attention to detail and
displaying intuitive knowledge of the functionality of their device.\\
\\
The greatest personal achievement of mine was deploying the first release of the
automation scripts I wrote. They controlled the attenuation equipment one of our
engineers was trained to operate. Previously he spent much of his time running
tests that had a lot of interaction with a computer application that changed
signal attenuation throughout a test. Because of my work, he was able to simply
configure a test, start it, and look at the results when it completed three or
more hours later. He was extremely grateful and I
felt like I had really made a difference for the first time in my Computer
Science career. This made it a joy to provide him with new features and maintain
these libraries and scripts later on. He was always very understanding when bugs
were discovered, and offered great feedback when I had written patches.

\section*{More About the Azimuth Libraries}
As mentioned above, I was able to provide automation to one of our engineers
that cut down on very tedious work. The development process, however, was an
interesting one. When the idea was first presented to me I was simply told ``We
think this can be done in Python. Figure it out.''\\
\\
A PDF came with the equipment that had a ``Programmer's Guide'' with a few
commands that a \emph{programmer} can issue. I knew that all I needed to do was
set the attenuation, and possibly check what it's currently set at. I found the
commands that did this, but I had no idea how to issue them.\\
\\
I discovered that this API was designed for TCL, which is just another scripting
language. I had never seen or used TCL before, but I kept at it and got these
commands to work. I ended up writing simple scripts that could be called from
the command line. This allowed us to connect to the control PC that's attached
to the Azimuth equipment and run the necessary commands. I wrote the Python
libraries around this concept.\\
\\
The way the network infrastructure is set up at Allion, combined with the 
libraries already being used in testing, it worked well to import the library
I had written for attenuation control and write tests that controlled
attenuation via my simple API. I wrote some simple scripts that demonstrated
the fact that this actually worked in a reliable way and showed it to my mentor
and some engineers around the shop. I talked to them more about how this could
be used in future testing, and that's how the testing libraries were introduced
to me.\\
\\
From there I learned how we loaded tests into a queue, and how I could write a
compatible test script that could be loaded and automate the attenuation
controls. A lot of work was involved, because things always seem to change once
all of the pieces are working together in a test. But after a lot of
communication and (sometimes frustrating) debugging, I had reliable test scripts
written that are still being used to this day.

\section*{More About Testing a New Product}
Due to the top secret nature of the project I was the test developer for, I have
to explain this with care. We wrote a test plan before we ever saw this device
or even knew what it was supposed to do. In fact, throughout the project we
barely ever discovered what its use cases would be. However, the device ran an
SSH server and had a serial connection, needed wireless testing, and was
expected to be compliant on a range of wireless access point manufacturers.\\
\\
To start I wrote up some mock scripts that simulated how the tests would
run were the device simply a computer running Linux. These tests included
putting the device into a suspend state, waking the device from a
suspend state, and using pings to verify that the device is in the state it
should be. Once I had worked out the kinks in developing these simple scripts
with a Linux computer, the only thing that needed to change was how the state
changes were implemented. Using SSH to connect would be the same, as would 
checking the state with pings.\\
\\
Because these scripts were ready to go before we ever saw the product, we were
able to present results within the first day of the customer coming to our shop
to present it to us. This got us off on the right foot in their eye, and opened
a channel of communication between me and the company representative - since I
was actively working with their device and finding out how things worked in
order to run these tests.\\
\\
One of the biggest challenges of this project was constant change of 
requirements. They appreciated my hard work and my detailed reports, but they
continually changed priority of what was being tested. For example, our test
plans seem to always focus on performance and we never covered that with this
project. We spent the majority of the effort on interoperability - which is
basically testing compatibility with a very wide range of AP/Router models.\\
\\
The most significant example of a requirements change that took place several
times was with a test where we were suspending the device and attempting to wake
it up after a certain amount of time had passed. This can be tricky because the
device isn't doing any communication while in a suspended state. Not only that,
but the company had developed their own implementation of the wakeup signal.
The scripts I wrote for this project went into a version-controlled repository,
and the number of revisions that have been committed is well over 200 now.
Each time the device was updated or test expectations changed, communication
was key in getting future tests to run smoothly. Hundreds of emails were sent,
there was a lot of interaction via their bug database, and multiple conference
calls took place. I worked with a fantastic team of engineers that were running
my scripts and reporting bugs, and many of the issues we ran into were solved
only because we did it as a team.

\section*{In Conclusion}
The thing I learned most about during this internship is communication. Within a
team, with a supervisor, or with a customer it is extremely important to
communicate well. On the team level it is vital that everyone is
on the same page. We had to work in stride without stepping on each other's
toes. We used things like collaboration software to stay in sync with technical
info such as who is running what test, how our equipment is configured, which
wireless channel we're using, and what tests still need to be done. We help
each other stay up to date on information about software updates and changes to
the test methods. Sometimes setting up for a test or running a test doesn't
quite work, and we help each other fix it.\\
\\
Communication with my mentor has been extremely rewarding. When I got here I
knew nothing about wireless technology or how it is tested. I have learned so
much from his demonstrations, and he does a great job of illustrating his points
with dry erase pens on the windows of the labs. Once I feel like I've met the
requirements of a test method, I run it by him again to double check and we talk
about the results afterward. Many times doing so has allowed us to improve the
test method so that we can provide even better results.\\
\\
When given the chance to talk directly to a customer about the tests or their
product, I've practiced a very diplomatic reserve. There have been features and
bugs that have frustrated me greatly and I have to simply observe and report.
What I really want to do is tell them they're doing it wrong, but that would
obviously not be constructive.\\
\\
Allion's scripts and libraries are written in Python. I had a growing passion
for working with Python before I started this internship, but I did not have a
very broad knowledge of the language. Throughout this internship I have been
constantly learning new ways of implementing tests and writing Python packages.
Russell, the senior software developer at the shop, has given me some excellent
instruction. His passive yet encouraging feedback is always helpful. He doesn't
ever come out and tell me I'm doing something wrong, but he's introduced a lot
of new concepts to me. Mimicking his style has helped me grow, and I feel
confident about the deployment and documentation of my work.\\
\\
The primary goal I set for myself was to learn about software development on a
larger scale, and to learn how a software developer interacts with fellow
developers and customers. The scale of our projects wasn't very large, but I
have at least worked on some development within a team. The goal that I feel I
am still pursuing is to do development of a software product for a customer. The
development I did here was scripts that the engineers and I were running. The
meta product that was delivered to the customer was a byproduct of those
scripts. While I appreciate everything I've learned here, the experiences I've
had, and the people I've gotten to work with, I still feel like there is much
more for me to learn.\\
\\
As mentioned before, there are two main benefits that I think Allion gained
from my work. The libraries I wrote and automation I developed will help with
many, many projects down the line. I really hope that when I'm not around the
code can be maintained without too much trouble. I've attempted to write as much
documentation as I could to help with that. The other benefit is the testing we've
been able to do for these projects. I've shown that I can get things working in
a pinch and really impress the customer. More importantly, perhaps, is the
effort I've given to communicating results to the customer. With the project I
spent the most work on, I really developed a relationship with the company that
allows good communication between our engineers and theirs. Our contract with
that company has
been extended multiple times, and while that definitely has to do with our
management's abilities and a lot of really good engineers, I do feel that I
played a part in that as well. I really began to feel this way when the
representative from their company specifically requested that I continue to
work on the project past one of the extension points.

\section*{Buzz Words}
The following is in addition to terminology associated with Linux and 
networking.
\begin{itemize}
\item DUT - Device Under Test
\item AP - Access Point
\item RVR - Rate vs. Range
\item OTA - Over the Air
\item Cops - Continuous Operations
\item ND\&S - Network Detection \& Selection
\item Cell Center/Middle/Edge
\end{itemize}

\end{document}
